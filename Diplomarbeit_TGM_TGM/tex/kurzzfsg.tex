\chapter*{Kurzfassung}

Das Thema der vorliegenden Diplomarbeit ist die Nutzung der Technik Augmented Reality beim Einsatz einer mobilen Applikation f{\"u}r den Auftraggeber 4relation Consulting GmbH.
Augmented Reality ist ein besonders aktuelles Thema, wie durch die im Moment bevorstehenden Ank{\"u}ndigungen der generellen Verf{\"u}gbarkeit von e.g. Google Glasses dokumentiert wird. Durch die gesteigerte Leistungsf{\"a}higkeit der f{\"u}r Mobile Phones und verwandte Ger{\"a}te, wie Smartphones, Tablets, etc., eingesetzten modernen Prozessoren scheint die Zeit reif f{\"u}r diese fortgeschrittenen Anwendungen.

Der Auftraggeber empfahl uns die Metaio SDK als Augmented Reality Technologie zu verwenden. Diese SDK bietet die M{\"o}glichkeiten sowohl 2D als auch 3D Objekte zu erkennen. Dies wird in unserer Android Applikation (App) implementiert. Ziel der App ist es durch das Trackingsystem verschiedene Fahrzeugmodelle zu erkennen und dem User Informationen zu diesen anzuzeigen. Diese Informationen werden in einer NAV Datenbank abgespeichert. Nach erfolgreicher Fahrzeug-Erkennung werden die entsprechenden Daten heruntergeladen und am Display des jeweiligen Ger{\"a}tes angezeigt.\\
Um solch eine Applikation umzusetzen werden Wissen und Technologien aus verschiedensten Teilen der Informatik ben{\"o}tigt. Unser Auftraggeber wird diese mobile Applikation zusammen mit in zentralen MS-Dynamics-NAV Servern ge-speicherten Informationen ihren Kunden vorstellen. Dieses Anwendungsszenario wurde im Rahmen dieser Diplomarbeit entwickelt und implementiert.