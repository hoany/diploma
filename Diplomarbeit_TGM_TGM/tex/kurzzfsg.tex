\chapter*{Kurzfassung}

Die vorliegende Diplomarbeit befasst sich mit der Technologie Augmented Reality, die in einer Mobil Applikation verwendet wird. Momentan ist die Technologie Augmented Reality im IT Markt sehr stark Nachgefragt. In der Zukunft wird die Technologie auch in mobil Anwendungen implementiert. Die Arbeit wird in Kooperation mit einer wichtig gestellten Industrie namens 4relation Consulting GmbH durchgef{\"u}hrt. Die Modernen Smartphones,  sind die Ger{\"a}te  auf denen man Augmented Reality Anwendung nutzen kann. In Rahmen dieser Diplomarbeit wird beschrieben wie man Augmented Reailty mit der Metaio SDK f{\"u}r eine Mobile Anwendung verwenden kann. Des Weiteren bieten Mobileger{\"a}te  immer mehr Grafik -- und Rechenleistung und haben damit das Potential einer nahezu idealen AR-Plattform. Die Mobile Ger{\"a}te haben nur einen Bruchteil der Leistung eines Desktop -- PC's, aber dennoch gen{\"u}gend, um verschiedene Trackingverfahren durchzuf{\"u}hren. In dieser Arbeit wird beschrieben wie man ein Trackingsystem anwendet um ein Objekt zu erkennen. Ziel ist es von diesem Projekt ,dass man einen Fahrzeug durch das Trackingsystem erkennt. Die mobil Applikation soll ein Fahrzeugmodell erkennen und nachher die Information vom jeweiligem Auto am Display ausgeben. Um solch eine Applikation umzusetzen werden Wissen und Technologien aus verschiedensten Teilen der Informatik ben{\"o}tigt. F{\"u}r das Trackingsystem wird die Logik von der Metaio GmbH verwendet. Diese Firma bietet eine SDK an um solche Trackingverfahren in einer mobilen Anwendung einzubauen.