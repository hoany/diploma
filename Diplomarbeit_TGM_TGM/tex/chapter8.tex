% Chapter8
\chapter{Future enhancements} \label{chapter:Future enhancement and possibilities}

The developed application can provide several possibilites for the future.
\subsection{Support for other platforms}
The Metaio SDK is available for several platforms.  \\It currently supports:
\begin{itemize}
\item Android
\item iOS
\item Windows(PC)
\item Unity
\end{itemize}
 \cite{metaioPlatforms} 
        \\
        \\
As described in chapter 6 ''Design Concept'' the application was implemented to be platform independent as possible.To run the application on another platform it is only necessary to swap the dependent Java code which access the Metaio SDK.For example it is possible to change the dependent code with C code to provide an app for the platform iOS. 


\newpage
\subsection{Support for Google Glass}
The Metaio SDK provides support for Google Glass if the operation system is Android with the version 4.4.2 .\cite{metaioGlass}\\
If the mobile app will be adapted to Google Glass in the future the user controls need to be updated. It is necessary because the current application  uses the touchscreen of the mobile phone  for user input and Google Glass uses speech recognition. Speech recognition is a technology to translate spoken words into text.    
More information about Google Glass can be found in  chapter 1 ''Google Glasses''.   
\subsection{Real estate}
The application currently is used to track cars.However Metaio also provides the functionality to track objects with the size of a building. Another enhancement for the future could be to track buildings instead of a car.In this scenario the application is used for the real estate market.So every person who walks by a property can simply put there mobile phone out and track the building to get information. Such as the price, number of rooms and so on.    
  
\newpage